%% start of file `template.tex'.
%% Copyright 2006-2013 Xavier Danaux (xdanaux@gmail.com).
%
% This work may be distributed and/or modified under the
% conditions of the LaTeX Project Public License version 1.3c,
% available at http://www.latex-project.org/lppl/.


\documentclass[11pt,letterpaper, sans]{article}     
%\usepackage[letterpaper, left=0.7in, right=0.7in,top=0.5in, bottom=0.5in]{geometry} 
\usepackage[top=0.5in, bottom=0.5in, left=0.8in, right=0.8in]{geometry}  

\usepackage{xcolor}

\usepackage{xeCJK}
\setCJKsansfont{Heiti SC}
\setCJKmainfont[BoldFont=Heiti SC Medium]{Heiti SC Light}
\renewcommand\familydefault{\sfdefault} 

\usepackage[T1]{fontenc}
\usepackage[none]{hyphenat}
\pagenumbering{gobble}

\usepackage{enumitem}
\usepackage{tabularx}


\newcommand{\newsec}[1]{\subsection*{\hspace{-1.5pt}\Large\uppercase{\bf #1}}}
\newcommand{\newsubsec}[1]{\vspace{0.2cm}{\bf #1}\vspace{0.1cm}} 

\newcommand{\itm}[1]{$\circ\;$ #1 \vspace{0.1em}\\}

\newenvironment{boxhonor}[1]
{
\def\temp{#1}
\begin{tabular}
{p{0.025\textwidth}>{\raggedright\arraybackslash}
p{0.85\textwidth}>{\raggedleft\arraybackslash}
p{0.125\textwidth}}
$\circ\;\,$ &
}
{
&  \temp \end{tabular}
}

% Make lists without bullets
\newenvironment{itemnew}{
  \begin{itemize}[label=$\circ$,leftmargin=*]{
  }
}{
  \end{itemize}
}







% personal data
\def\name{林義軒, PhD}

\def\header{
\begin{center}
{\LARGE\bf \name} \vspace{0.3cm} \\
yi-hsuan.lin@outlook.com
| $+$1-416-829-6531 | 
\href{https://www.linkedin.com/in/yihsuanlinphysics}{www.linkedin.com/in/yihsuanlinphysics}
%\href{http://individual.utoronto.ca/yihsuanlin}{individual.utoronto.ca/yihsuanlin} \\
%\href{https://laphysique.github.io}{laphysique.github.io}
\end{center}
}



\usepackage{hyperref}
\hypersetup{
    colorlinks = true,
    urlcolor = black,
    linkcolor = black,
    citecolor = black,
  pdfauthor = {\name},
  pdftitle = {\name: resume},
  pdfsubject = {resume},
  pdfpagemode = UseNone
}


%----------------------------------------------------------------------------------
%            content
%----------------------------------------------------------------------------------
\begin{document}

% Header
\header

\begin{flushleft}

%\newsec{Summary}

%A theoretical physics PhD with excellent skills of numerical analysis and quantitative modeling, having 10+ years of experience in solving quantitative complex problems %by numerical analysis and quantitative modeling 
%into your team. 
%the~\team~at
%~\company.
%\vspace{-2.5em}\ \



\newsec{經歷摘要}

\begin{itemize}[leftmargin=*]\itemsep-0.2em
\item 精通理論及計算生物物理與生物資訊,超過十年學術及產業界經驗。
\item {\bf 14} 篇SCI論文,總引用數超過{\bf 700}次。
\item 曾獲邀國際知名學術機構及國際會議進行學術演講{\bf 10}次。
\item Github編程檔案庫:\href{https://github.com/laphysique}{github.com/laphysique} 
\item 學術網站:\href{http://individual.utoronto.ca/yihsuanlin}{individual.utoronto.ca/yihsuanlin}
\end{itemize}

\newsec{教育背景}

美國俄亥俄州立大學{\bf 物理學博士} \hfill 2015 \\
美國伊利諾大學香檳分校{\bf 物理學學士} (GPA 3.74)  \hfill 2009

\newsubsec{學程證書}: 

金融工程與風險管理 (Coursera.org / 美國哥倫比亞大學) \hfill 2020 \\
基礎深度學習榮譽學程 (Coursera.org / 莫斯科國立高等經濟學院)  \hfill 2020 \\
貝氏統計機器學習榮譽學程 (Coursera.org / 莫斯科國立高等經濟學院)  \hfill 2020 \\
強化學習應用榮譽學程 (Coursera.org / 莫斯科國立高等經濟學院)  \hfill 2020 \\


\newsec{工作經歷}

{\bf HTuO Biosciences分子建模首席科學家} 
\hfill 2021年一月至今
\vspace{-0.5em} \\
\begin{itemize}[leftmargin=*]\itemsep-0.2em
\item 開發分子動力學模擬力場
\item 結合機器學習優化仿真模擬性能
\item 實施數學物理學以驗證仿真模擬方法的穩定性
\end{itemize}

%{\bf Freelance ESG Data Scientist} \hfill Jun 2020 -- present
%\vspace{-0.5em}\\
%\begin{itemize}[leftmargin=*]\itemsep-0.1em
%\item Applied supervised machine learning algorithms and Bayesian statistics to build models for time series
%forecasting of environmental, social and corporate governance (ESG) financial data.
%\end{itemize}

%{\bf Science Advisor, StemCellerant} \hfill Nov 2019 -- Dec 2019 
%\vspace{-0.5em}\\
%\begin{itemize}[leftmargin=*]\itemsep-0.1em
%\item Provided consultation on biotech application and business development of new stem cell differentiation technology and systems biology
%\end{itemize}

\vspace{-1.5em}\ \

{\bf 多倫多大學暨多倫多兒童醫院博士後研究員} \hfill 2015年七月至2021年七月
\vspace{-0.5em}\\
\begin{itemize}[leftmargin=*]\itemsep-0.2em
\item 發表了12篇物理、化學與生物交叉學科理論及計算方法的SCI碖文
\item 指導了4位初級科學家(研究生和本科生)
\item 研究項目:生物分子的液--液相分離理論
\end{itemize}

\newsec{工作技能}
\begin{itemize}[leftmargin=*]\itemsep-0.1em
%
\item {\bf 數學與統計學:}
數據分析、
貝氏統計學、
多變量線性與非線性優化、
線性代數、
多變量微積分、
隨機微積分、
常微分與偏微分方程、
複變分析
\item{\bf 科學建模:}
理論物理、分子生物物理、生物資訊、蒙地卡羅模擬、分子動力模擬、統計學建模、機器學習、深度學習、強化學習、資料視覺化、主值分析、時間序列預測
%
\item{\bf 編程語言:} 
Python, Matlab/Octave, C/C++, Mathematica, Julia, SQL
%
\item{\bf 編程工具:}
Numpy, Scipy, Pandas, Matplotlib, Scikit-Learn, PyMC, TensorFlow, PyTorch, SQLite, MPI, Git, PyCharm
%

\end{itemize}

%\vspace{-1em}

\newsec{榮譽獎項}
\begin{itemize}[leftmargin=*]\itemsep-0.1em
\item 美國生物物理學會無序蛋白組年度博士後研究員獎項 \hfill 2019 
\item 台灣教育部出國留學獎學金 \hfill 2007--2013 
\item 第三十六屆國際物理奧林匹亞金牌\hfill 2005
\end{itemize}


\end{flushleft}

\end{document}

%% end of file `template.tex'.
