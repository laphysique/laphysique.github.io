%% start of file `template.tex'.
%% Copyright 2006-2013 Xavier Danaux (xdanaux@gmail.com).
%
% This work may be distributed and/or modified under the
% conditions of the LaTeX Project Public License version 1.3c,
% available at http://www.latex-project.org/lppl/.


\documentclass[11pt,letterpaper, sans]{article}     
%\usepackage[letterpaper, left=0.7in, right=0.7in,top=0.5in, bottom=0.5in]{geometry} 
\usepackage[top=0.5in, bottom=0.5in, left=0.8in, right=0.8in]{geometry}  
\usepackage{xcolor}
\renewcommand\familydefault{\sfdefault} 
\usepackage[T1]{fontenc}
\usepackage[none]{hyphenat}
\pagenumbering{gobble}



\usepackage{enumitem}
\usepackage{tabularx}





\newcommand{\newsec}[1]{\subsection*{\hspace{-1.5pt}\uppercase{#1}}}
\newcommand{\newsubsec}[1]{\vspace{0.2cm}{\bf #1}\vspace{0.1cm}} 

\newcommand{\itm}[1]{$\circ\;$ #1 \vspace{0.1em}\\}

\newenvironment{boxhonor}[1]
{
\def\temp{#1}
\begin{tabular}
{p{0.025\textwidth}>{\raggedright\arraybackslash}
p{0.85\textwidth}>{\raggedleft\arraybackslash}
p{0.125\textwidth}}
$\circ\;\,$ &
}
{
&  \temp \end{tabular}
}

% Make lists without bullets
\newenvironment{itemnew}{
  \begin{itemize}[label=$\circ$,leftmargin=*]{
  }
}{
  \end{itemize}
}







% personal data
\def\name{Yi-Hsuan Lin, PhD}

\def\header{
\begin{center}
{\LARGE\bf \name} \vspace{0.3cm} \\
yi-hsuan.lin@outlook.com
| $+$1-416-829-6531 | 
\href{https://www.linkedin.com/in/yihsuanlinphysics}{www.linkedin.com/in/yihsuanlinphysics}
%\href{http://individual.utoronto.ca/yihsuanlin}{individual.utoronto.ca/yihsuanlin} \\
%\href{https://laphysique.github.io}{laphysique.github.io}
\end{center}
}



\usepackage{hyperref}
\hypersetup{
    colorlinks = true,
    urlcolor = black,
    linkcolor = black,
    citecolor = black,
  pdfauthor = {\name},
  pdftitle = {\name: resume},
  pdfsubject = {resume},
  pdfpagemode = UseNone
}


%----------------------------------------------------------------------------------
%            content
%----------------------------------------------------------------------------------
\begin{document}

% Header
\header

\begin{flushleft}

%\newsec{Summary}

%A theoretical physics PhD with excellent skills of numerical analysis and quantitative modeling, having 10+ years of experience in solving quantitative complex problems %by numerical analysis and quantitative modeling 
%into your team. 
%the~\team~at
%~\company.
%\vspace{-2.5em}\ \

\newsec{Highlights}

\begin{itemize}[leftmargin=*]\itemsep-0.2em
\item {\bf 10+} years of research experience in theoretical and computational biophysics and bioinformatics
\item {\bf 15} scientific papers in peer-reviewed journals
cumulatively cited {\bf over 700} times
\item {\bf 11} invited seminars and colloquia in world-leading academic institutes and conferences
%\item Co-chaired a special session %the ``Using Polymer Sequence to Control Material Properties" 
%at the 2019 American Physical Society March Meeting (Mar 8, 2019)
\item GitHub repositories: \href{https://github.com/laphysique}{github.com/laphysique} 
\item Academia website:  \href{http://individual.utoronto.ca/yihsuanlin}{individual.utoronto.ca/yihsuanlin}
\end{itemize}

\newsec{Education}

{\bf Ph.D., Physics}, The Ohio State University, Ohio, USA \hfill 2015 \\
{\bf B.Sc., Physics}, University of Illinois at Urbana-Champaign, Illinois, USA (GPA 3.74) \hfill 2009

\newsubsec{Certificates}: 

Financial Engineering and Risk Management I \& II (Columbia Univ) \hfill 2020 \\
Introduction to Deep Learning with Honors (Coursera.org/HSE Univ)  \hfill 2020 \\
Bayesian Methods for Machine Learning with Honors (Coursera.org/HSE Univ)  \hfill 2020 \\
Practical Reinforcement Learning with Honors (Coursera.org/HSE Univ)  \hfill 2020 \\

\newsec{Experience}

{\bf Molecular Modeling Lead,
HTuO Biosciences} 
\hfill Jan 2021 -- present
\vspace{-0.5em} \\
\begin{itemize}[leftmargin=*]\itemsep-0.2em
\item Developing fundamental physics frameworks of molecular dynamics simulation force fields
\item Incorporating machine learning to parametrize force fields and optimize their simulation performance
\item Implementing mathematical physics to validate stability of simulation methods
\end{itemize}

%{\bf Freelance ESG Data Scientist} \hfill Jun 2020 -- present
%\vspace{-0.5em}\\
%\begin{itemize}[leftmargin=*]\itemsep-0.1em
%\item Applied supervised machine learning algorithms and Bayesian statistics to build models for time series
%forecasting of environmental, social and corporate governance (ESG) financial data.
%\end{itemize}

%{\bf Science Advisor, StemCellerant} \hfill Nov 2019 -- Dec 2019 
%\vspace{-0.5em}\\
%\begin{itemize}[leftmargin=*]\itemsep-0.1em
%\item Provided consultation on biotech application and business development of new stem cell differentiation technology and systems biology
%\end{itemize}

\vspace{-1.5em}\ \

{\bf Postdoctoral Fellow, 
University of Toronto / Hospital for Sick Children} 
\hfill Jul 2015 -- Jul 2021  %\vspace{0.2em}
\vspace{-0.5em}\\
%Project: {\it Theories for sequence-dependent phase behaviors of biomolecular condensates}\vspace{-0.2em} 
\begin{itemize}[leftmargin=*]\itemsep-0.2em
\item Published {\bf 12} peer-reviewed papers in theoretical/computational physics, chemistry, and biology
\item Supervised and mentored over {\bf 4} junior scientists (graduate students and trainees)
%\item Conduct research on theoretical and computational biophysics and biochemistry 
%\item Develop quantitative models for predicting complex protein and nucleic acid behavior%biological phenomena
%\item Develop Monte Carlo simulation programs from scratch for protein assemblies 
\item Project: {\it Theories for sequence-dependent phase behaviors of biomolecular condensates}
\end{itemize}

\vspace{-1.5em}\ \

{\bf Graduate Research Associate, %in Biophysics, 
The Ohio State University}  \hfill Jul 2012 -- May 2015
\vspace{-0.5em} \\
%Project: {\it Biophysics of interactions between proteins and nucleic acids} \vspace{-0.2em} 
\begin{itemize}[leftmargin=*]\itemsep-0.2em
%\item Developed statistical physics models for DNA and RNA behavior
\item Published {\bf 2} peer-reviewed papers in theoretical physics and bioinformatics
\item Established theoretical framework for online RNA-protein binding predictor \href{http://bioserv.mps.ohio-state.edu/RBPBind/}{\underline{\bf RBPBind}}
\item Project: {\it Biophysics of interactions between proteins and nucleic acids}
%\item Conducted statistical analysis for RNA and microRNA databases
%\item Conducted Monte Carlo simulation for RNA behavior
\end{itemize}

%\noindent
%{\bf Graduate Research Associate in Quantum Physics,
%The Ohio State University}  \hfill Jul 2011 -- May 2012
%\vspace{-0.5em} \\
%\begin{itemize}[leftmargin=*]\itemsep-0.1em
%\item Applied quantum physics methods to study the behavior of atoms at absolute zero degree.
%\item Project: {\it BEC-BCS crossover in cold-atomic systems}
%\end{itemize}


\newsec{Skills}
\begin{itemize}[leftmargin=*]\itemsep-0.1em
%
\item {\bf Math/Stat:}
Numerical Analysis, 
Bayesian Statistics, 
Multivariate Linear/Nonlinear Optimization, 
%\item {\bf Advanced Math for Physics:} 
Linear Algebra, 
Multivariable Calculus,
Stochastic Calculus,
Partial Differential Equation, 
%Fourier and Laplace Transformation
Complex Analysis %, Functional Optimization %with Functional Analysis
\item{\bf Modelings:}
Theoretical Physics, Molecular Biophysics, Bioinformatics,
Monte Carlo Simulation, Molecular Dynamics Simulation, %Support-Vector Machines, 
%Neural Networks, Random Forests,
Data-Driven Statistical Modeling,
Machine Learning,
%Natural Language Processing, 
%Dynamic Programming,
Deep Learning, Reinforcement Learning, %Gaussian Mixture Model,
Data Visualization, 
Principal Component Analysis, 
Time Series Forecasting
%
\item{\bf Programming:} 
Python, Matlab/Octave, C/C++, Mathematica, Julia, SQL
%
\item{\bf Tools:}
Numpy, Scipy, Pandas, Matplotlib, Scikit-Learn, PyMC, TensorFlow, PyTorch, SQLite, MPI, Git
%


\end{itemize}

%\vspace{-1em}

\newsec{Honors and Awards}

\noindent
{\bf Postdoctoral Award}, Intrinsically Disordered Protein Subgroup, Biophysical Society (USA) \hfill 2019 \\
{\bf Scholarship for Study Abroad}, Taiwan Ministry of Education  \hfill 2007--2013 \\
{\bf Gold Medal}, The 36$^{ \mathsf{th}}$ International Physics Olympiad \hfill 2005


\end{flushleft}

\end{document}

%% end of file `template.tex'.
