\documentclass[11pt]{../yhlcv}     

\def\tname#1{{\bf #1}}


%----------------------------------------------------------------------------------
%            content
%----------------------------------------------------------------------------------
\begin{document}

\title{Resume}
\name{Yi-Hsuan Lin, PhD}
\mobile{416-829-6531}
\email{yi-hsuan.lin@outlook.com}
\linkedin{yihsuanlinphysics}
\github{laphysique}

\maketitle

\raggedright

\section*{Highlights}

\begin{itemize}[leftmargin=*]\itemsep-0.2em
\item {\bf 10+ years} of research experience in {\bf theoretical biophysics}, {\bf computational biology}, and {\bf bioinformatics}
\item {\bf 10+ years} of coding experience in {\bf C/C++} and {\bf Python} for scientific computing
\item {\bf 17} scientific papers in peer-reviewed journals cumulatively cited {\bf over 1100} times
%\item {\bf 11} invited seminars and colloquia in world-leading academic institutes and conferences, including the {\bf Machine Learning in Biophysics} lecture at the 2021 Canadian Association for Physicists Congress.
\end{itemize}

\section*{Education}

{\bf Ph.D., Physics}, The Ohio State University, Ohio, USA \hfill 2015 \\
{\bf B.Sc., Physics}, University of Illinois at Urbana-Champaign, Illinois, USA (GPA 3.74, High Distinction) \hfill 2009

\subsection*{Certificates}
Introduction to Deep Learning with Honors (Coursera.org/HSE Univ) \\
Bayesian Methods for Machine Learning with Honors (Coursera.org/HSE Univ)   \\
Practical Reinforcement Learning with Honors (Coursera.org/HSE Univ)  \\
Financial Engineering and Risk Management I \& II (Coursera.org/Columbia Univ)

\section*{Experience}

{\bf Molecular Modelling Lead}  \hfill Jan 2021 -- present \vspace{0.25em} \\
HTuO Biosciences, Vancouver, BC, Canada 
\vspace{-0.7em} \\
\begin{itemize}[leftmargin=*]\itemsep-0.2em
\item Design the computational biophysics methods for proprietary computer-aided drug design (CADD) platform
\item Establish the architecture and contents of the CADD platform via C, Python, and Cython
%
\item Incorporating machine learning methods to optimize platform performance
%
\item Attended due diligence meetings with prospective investors to provide scientific consultation.
\end{itemize}

{\bf Data Scientist} \hfill Jun 2020 -- Oct 2020 \vspace{0.25em} \\
%\href{https://sustainability.exchange}{
Sustainability.Exchange, Toronto, ON, Canada
\vspace{-0.7em}\\
\begin{itemize}[leftmargin=*]\itemsep-0.2em
\item Applied supervised machine learning algorithms and Bayesian statistics to build models for time series
forecasting of environmental, social and corporate governance (ESG) financial data.
\end{itemize}

{\bf Postdoctoral Fellow} \hfill Jul 2015 -- Jul 2021 \vspace{0.25em} \\
University of Toronto \& Hospital for Sick Children, Toronto, ON, Canada
\vspace{-0.7em}\\
\begin{itemize}[leftmargin=*]\itemsep-0.2em
\item Developing theoretical and computational methods for investigating biological liquid-liquid phase separation
\item Collaborating with experimentalists and computational biologists to test the above-mentioned physics theory
%\item Published {\bf 13} peer-reviewed papers in theoretical/computational physics, chemistry, and biology
%\item Supervised and mentored over {\bf 3} junior scientists (students and trainees)
\item Published peer-reviewed papers and mentored junior scientists 
\item Project: {\it Theories for sequence-dependent phase behaviors of biomolecular condensates}
\end{itemize}

\section*{Skills}
\begin{itemize}[leftmargin=*]\itemsep-0.1em
%
\item{\bf Programming:} 
C/C++, Python, Cython, Matlab/Octave, Mathematica, Julia, SQL
%
\item{\bf Tools:}
Numpy, Scipy, Pandas, Matplotlib, Scikit-Learn, PyMC, SQLite, MPI, PyCharm, Git

%
\item {\bf Math/Stat:}
Numerical Analysis, 
Bayesian Statistics, 
Multivariate Linear/Nonlinear Optimization, 
%\item {\bf Advanced Math for Physics:} 
Linear Algebra, 
Multivariable Calculus,
Stochastic Calculus,
Partial Differential Equation, 
%Fourier and Laplace Transformation
Complex Analysis %, Functional Optimization %with Functional Analysis
\item{\bf Modelings:}
Theoretical Physics, Molecular Biophysics, Bioinformatics,
Monte Carlo Simulation, Molecular Dynamics Simulation, %Support-Vector Machines, 
%Neural Networks, Random Forests,
Data-Driven Statistical Modeling,
Machine Learning,
%Natural Language Processing, 
%Dynamic Programming,
Deep Learning, %Reinforcement Learning, %Gaussian Mixture Model,
Data Visualization, 
Principal Component Analysis, 
Time Series Forecasting
%
\end{itemize}

\section*{Publications}

\begin{etaremune}[leftmargin=0.26in]\itemsep-0.2pt

\item
Gaither J$^*$, \tname{Lin Y-H}$^*$, and Bundschuh R (2022)
RBPBind: quantitative prediction of protein-RNA interactions
{\it J Mol Biol} {\bf 434} 167515
($^*$equal contribution)

\item 
\tname{Lin Y-H}, Wu H, Jia B, Zhang M, and Chan HS (2022)
Assembly of model postsynaptic densities involves interactions auxiliary to stoichiometric binding
{\it Biophys J} {\bf 121} 151--171

\item 
Wessén J, Pal T, Das S, \tname{Lin Y-H}, and Chan HS (2021)
A simple explicit-solvent model of polyampholyte phase behaviors and its ramifications for dielectric effects in biomolecular condensates.
{\it J Phys Chem B} {\bf 125} 4337--4358 %(selected supplementary cover)

\item 
Das S, \tname{Lin Y-H}, Vernon RM, Forman-Kay JD, and Chan HS (2020)
Comparative roles of charge, $\pi$, and hydrophobic interactions in sequence-dependent phase separation of intrinsically disordered proteins.
{\it Proc Natl Acad Sci USA} {\bf 117} 28795--28805

\item 
Amin AN$^*$, \tname{Lin Y-H}$^*$, Das S, and Chan HS (2020)
Analytical theory for sequence-specific binary fuzzy complexes of charged intrinsically disordered proteins.
{\it J Phys Chem B} 
{\bf 124} 6709--6720
($^*$equal contribution) %, selected supplementary cover) 

\item
\tname{Lin Y-H}, Brady JP, Chan HS, and Ghosh K (2020)
A unified analytical theory of heteropolymers for sequence-specific phase behaviors of polyelectrolytes and polyampholytes. 
{\it J Chem Phys} {\bf 152} 045102

\item
Cinar H, Oliva R, \tname{Lin Y-H}, Chen X, Zhang M, Chan HS, and Winter RHA (2020)
Pressure sensitivity of SynGAP/PSD-95 condensates as a model for postsynaptic densities and its biophysical and neurological ramifications. 
{\it Chem Eur J} {\bf 26} 11024--11031 %(cover feature) 

\item 
Das S, Amin AN, \tname{Lin Y-H}, and Chan HS (2018)
Coarse-grained residue-based models of disordered protein condensates: 
utility and limitations of simple charge pattern parameters.
{\it Phys Chem Chem Phys} {\bf 20} 28558--28574 

\item
\tname{Lin Y-H}, Forman-Kay JD, and Chan HS (2018)
Theories for sequence-dependent phase behaviors of biomolecular condensates.
{\it Biochemistry} {\bf 57} 2499--2508

\item 
Das S, Eisen A, \tname{Lin Y-H}, and Chan HS (2018)
A lattice model of charge-pattern-dependent polyampholyte phase separation.
{\it J Phys Chem B} {\bf 122} 5418--5431

\item
\tname{Lin Y-H}, Brady JP, Forman-Kay JD, and Chan HS (2017) 
Charge pattern matching as a ``fuzzy'' mode of molecular recognition for the functional phase separations of intrinsically disordered proteins.
{\it New J Phys} {\bf 19} 115003

\item 
Brady JP, Farber PJ, Sekhar A, \tname{Lin Y-H}, Huang R, Bah A, Nott TJ, Chan HS, Baldwin AJ, Forman-Kay JD, and Kay LE (2017) 
Structural and hydrodynamic properties of an intrinsically disordered region of a germ-cell specific protein upon phase separation. 
{\it Proc Natl Acad Sci USA} {\bf 114} E8194--E8203

\item
\tname{Lin Y-H} and Chan HS (2017) 
Phase separation and single-chain compactness of charged disordered proteins are strongly correlated. 
{\it Biophys J} {\bf 112} 2043--2046

\item 
\tname{Lin Y-H}, Song J,Forman-Kay JD, and Chan HS  (2017) 
Random-phase-approximation theory for sequence-dependent, biologically functional liquid-liquid phase separation of intrinsically disordered proteins. 
{\it J Mol Liq} {\bf 228} 176--193

\item
\tname{Lin Y-H}, Forman-Kay JD, and Chan HS (2016) 
Sequence-specific polyampholyte phase separation in membraneless organelles. 
{\it Phys Rev Lett} {\bf 117} 178101

\item 
\tname{Lin Y-H} and Bundschuh R (2015) 
RNA structure generates natural cooperativity between single-stranded RNA binding proteins targeting 5' and 3'UTRs.
{\it Nucleic Acids Res} {\bf 43} 1160--1169

\item 
\tname{Lin Y-H} and Bundschuh R (2013) 
Interplay between single-stranded binding proteins on RNA secondary structure.
{\it Phys Rev E} {\bf 88} 052707

\end{etaremune} 

\section*{Honors and Awards}

\noindent
{\bf Postdoctoral Award}, Intrinsically Disordered Protein Subgroup, Biophysical Society (USA) \hfill 2019 \\
{\bf Connell Award for Postdoctoral Fellow}, Department of Biochemistry, University of Toronto \hfill 2018 \\
{\bf Scholarship for Study Abroad}, Taiwan Ministry of Education  \hfill 2007--2013 \\
{\bf Gold Medal}, The 36$^{ \mathsf{th}}$ International Physics Olympiad \hfill 2005






\end{document}


